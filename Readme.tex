\documentclass[a4paper, 12pt, oneside]{article}

\usepackage[utf8]{inputenc}
\usepackage[T1]{fontenc}
\usepackage{lmodern} % Enlève plein de warnings :D et permet le tt gras
\usepackage[french]{babel}

% \usepackage[usenames,dvipsnames,svgnames,table]{xcolor}

\usepackage{amsmath}
\usepackage{amssymb}
% \usepackage{mathrsfs}

\usepackage{hyperref}
\usepackage{scrextend}

\usepackage{graphicx}

\usepackage{color}
\usepackage{listings}
\definecolor{listinggray}{gray}{0.9}
\definecolor{lbcolor}{rgb}{0.95,0.95,0.95}
\lstset{
	backgroundcolor=\color{lbcolor},
	tabsize=4,
	rulecolor=,
	language=Java,
	basicstyle=\scriptsize\ttfamily,
	upquote=false,
	aboveskip={.5\baselineskip},
	columns=fixed,
	showstringspaces=false,
	extendedchars=true,
	showspaces=false,
	breaklines=true,
	prebreak = \raisebox{0ex}[0ex][0ex]{\ensuremath{\hookleftarrow}},
	frame=single,
	showtabs=false,
	identifierstyle=\ttfamily,
	keywordstyle=\bfseries\color[rgb]{0,0,1},
	commentstyle=\color[rgb]{0.133,0.545,0.133},
	stringstyle=\color[rgb]{0.627,0.126,0.941},
}

% ---------------
% Commmandes
% ---------------

% Argument obligatoire : nom du type agrégé
\newenvironment{typeag}[1][]{\noindent \textbf{type} \texttt{#1} \{\begin{addmargin}[2em]{0em}}{\end{addmargin}\}}
% Arguments obligatoires : Nom — Type — Description
\newcommand{\variable}[3]{\noindent \texttt{#1} : \textit{#2} --- #3}

% Algorithme informel % Ou \scriptsize
\newenvironment{algoinfo}{\begin{itemize}\small}{\end{itemize}}

% \renewcommand\arraystretch{1.2}

\newcommand{\oeuvre}[1]{\textit{#1}}

\newcommand{\lien}[2]{\noindent #1 :\\{\small\url{#2}}}

% ---------------
% Main
% ---------------
\title{Manuel du programme Démineur}
\author{Mohamed \bsc{Lakhal}\\Alexis \bsc{Cabodi}}
\date{Révision du\\\today}

\begin{document}
\maketitle
\newpage
\tableofcontents
\newpage

\section{Mode d'emploi}
Le but du jeu est de découvrir toutes les cases non minées en évitant de cliquer sur une case minée. Le joueur a la possibilité de poser un drapeau sur une case qu'il soupçonne contenir une mine. Attention, le drapeau peut aussi être posé sur une case quelconque.
\bigskip

Après le lancement de l'application, le joueur saisit divers paramètres afin d'initialiser sa partie : nombre de lignes, colonnes, mines.
À l'aide du clic gauche de la souris, le joueur peut découvrir une case. Avec le clic droit, il pose un drapeau sur la case.
\bigskip

Si le joueur clique par mégarde sur une mine, la partie s'arrête et le joueur a perdu.
Accessoirement, le joueur peut rejouer en gardant les mêmes paramètres et ainsi tenter de battre son temps.
\newpage
\section{Développement}
Cette section va traiter des divers aspects liés au développement de l'application.
\subsection{Résolution des problèmes}
Les problèmes rencontrés : 
\bigskip

- Types agrégés :   

\begin{typeag}[Case]
	\variable{visible}{booleen}{test si une case est découverte} \\
	\variable{drapeau}{booleen}{test si une case a un drapeau} \\
	\variable{contenu}{entier}{-1 si une mine sinon le nombre de mines adjacentes}
\end{typeag}

\begin{typeag}[Demineur]
	\variable{nl}{entier}{nombre de lignes} \\
	\variable{nc}{entier}{nombre de colonnes} \\
	\variable{nm}{entier}{nombre de mines} \\
	\variable{grille}{Case [nl][nc]}{représente la grille de jeu}
	decouverte = 0 : entier (stack le nombre de cases découvertes)
\subsection{Répartition des tâches}

\end{document}